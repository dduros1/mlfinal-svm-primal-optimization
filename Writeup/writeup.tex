\documentclass[letterpaper, 11pt]{article}
\usepackage{latexsym}
\usepackage{amssymb}
\usepackage{times}
\usepackage[margin=1in]{geometry}
\usepackage{amsmath,amsfonts,amsthm}
\usepackage{graphicx}
\usepackage{algorithm}
\usepackage[noend]{algorithmic}
\usepackage{cite}

\begin{document}

\title{Comparison of SVM Optimazation Techniques in the Primal}
\author{Diane Duros and Jonathan Katzman}
\maketitle
\begin{abstract}This paper examines the efficacy of different optimzation techniques in a primal formulation of a support vector machine (SVM).  Three main techniques are compared.  The benchmark to compare all three techniques was sentiment analysis on movie reviews.
\end{abstract}
                                                                                                                                                                                                                                                                           
\section{Introduction}


\section{Data}

\section{SVMs}

\subsection{Multiclass SVM}

\section{Primal Optimization Methods}

\subsection{Gradient Descent}

\subsection{Newton's Approximation}

\textbf{RBF Kernel}

Large sigma values produced poor results (on a pairwise SVM, accuracies were less than .5), but terminated quickly.  However, we did not have any terminating runs with small sigma values. <Hopefully until Monday night when we write more about this! TODO>


\subsection{Stochastic Subgradient}


\nocite{*}
\bibliographystyle{plain} 
\bibliography{writeup}

\end{document}